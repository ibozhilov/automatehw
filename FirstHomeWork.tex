\documentclass[12pt,a4paper]{amsart}

\usepackage[T1, T2A]{fontenc}
\usepackage[utf8]{inputenc}
\usepackage[american, bulgarian]{babel}
\usepackage[active]{srcltx} % SRC Specials
\usepackage{enumerate}

\vfuzz2pt % Don't report small over-full v-boxes

\theoremstyle{definition}
\newtheorem{zadd}{Задача}

\setlength{\textwidth}{18.5cm} \setlength{\textheight}{10in}
\setlength{\topmargin}{-0.5cm} \setlength{\voffset}{-0.5in}
\setlength{\hoffset}{-0.5in}
\parindent 0.5cm
\oddsidemargin 0in \evensidemargin 0in \pagestyle{empty}

\newfont{\bbb}{msbm10}
\def\ZZ{\mbox{\bbb Z}}
\def\NN{\mbox{\bbb N}}
\def\RR{\mbox{\bbb R}}
\def\CC{\mbox{\bbb C}}
\def\FF{\mbox{\bbb F}}
\def\QQ{\mbox{\bbb Q}}
\def\Im{\mathop{\rm Im}}
\def\Re{\mathop{\rm Re}}


\def\KONTROLNO{\bf\Large <*info*>}
\def\Variant{}
\def\Specialnost{<*spec*>}
\def\deadline{<*date*>}
\def\FN{\hfil{<*FN*>}\hfil}
\def\gr{\hfil{\bf <*gr*>}\hfil}
\def\potok{\hfil{\bf II}\hfil}
\def\kurs{\hfil{\bf 1}\hfil}
\def\name{\hfil{\bf <*name*>}\hfil}


\begin{document}
\thispagestyle{empty}

\centerline{\vbox{\offinterlineskip\halign{\vrule#&&\,#\,&\vrule#\cr
\multispan{13}\hrulefill\cr &вариант&&ф.
номер&&група&&поток&&курс&&\hfil специалност\hfil&&\cr
\multispan{13}\hrulefill\cr &\vrule width0pt depth6pt
height12pt\hfil{\bf\large \Variant}\hfil&&\FN\strut&&\gr
&&\potok&&\kurs&&\hfil\hspace{50pt}{\bf \Specialnost}\hspace{50pt}\hfil&&\cr
\multispan{13}\hrulefill\cr &\vrule width0pt depth6pt
height12pt\hfil Име:\hfil&&
\multispan{9}\hfil{\bf \name}\hfil&&\cr
\multispan{13}\hrulefill\cr}}}

\vspace{1cm} \centerline{\KONTROLNO}

\centerline{спец. \Specialnost}
%
\centerline{\deadline}

\begin{zadd}
Да се намерят базисите на сумата и сечението на $U$ и $V$, където:
\begin{equation*}
<*random_linear_system*>
\end{equation*}
\begin{equation*}
\text{ и }U\text{ е линейната обвивка на векторите:}\ \ 
<*random_vectors*>
\end{equation*}
\end{zadd}

\bigskip

\begin{zadd}
Да се реши матричното уравнение:
$$
\begin{pmatrix}1 & 1 & 1 & 1 & 1\cr 1 & 1 & 1 & 1 & 0\cr 1 & 1 & 1 & 0 & 0\cr 1 & 1 & 0 & 0 & 0\cr 1 & 0 & 0 & 0 & 0\end{pmatrix}.X.\begin{pmatrix}1 & 1 & 1 & 1 & 1\cr 0 & 1 & 1 & 1 & 1\cr 0 & 0 & 1 & 1 & 1\cr 0 & 0 & 0 & 1 & 1\cr 0 & 0 & 0 & 0 & 1\end{pmatrix}=<*random_matrix*>
$$
\end{zadd}

\bigskip

\begin{zadd}
Нека $\varphi \in Hom(\QQ^4) $, който спрямо даден базис има матрица А:
$$
A=<*random_matrix2*>
$$
намерете базиси на $Ker\varphi$ и $Im\varphi$.
\end{zadd}

\bigskip

\begin{zadd}
Да се намерят $1\text{-мерните}$ $\varphi\text{-инвариантните}$ подпространства на $\QQ^3$, ако линейният оператор $\varphi \in Hom(\QQ^3) $ има матрица А:
$$
<*random_matrix1*>
$$
спрямо даден базис.
\end{zadd}
\end{document}